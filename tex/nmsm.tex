% !TEX program=lualatex
\documentclass{gulartcl}
\usepackage{preamble}

\title{NMSM Homework Exercises}
\author{Guglielmo Bordin}
\date{\today}

\begin{document}
\maketitle
\section{Sampling random points within \emph{d}-dimensional domains by hit and
miss}
I skipped the integration on the rectangle, solving only the disk case.  The
source code is in \texttt{A01b\_disk\_hit\_miss.c}; I implemented the main part
of the algorithm like this:

\lstinputlisting[firstline=32, lastline=46, language=C]{../src/A01b_disk_hit_miss.c}

The error as a function of the number of throws is shown in \figref{fig:A01b}.
It is comfortably under \qty{1}{\percent} with around \numrange{25000}{30000}
iterations.

\begin{figure}
    \centering
    \includesvg[inkscapelatex=false]{img/A01b.svg}
    \caption{error in the Monte Carlo estimation of the area of a unit
        disk, as a function of the number of \textquote{throws}.}
    \label{fig:A01b}
\end{figure}

\section{Sampling random numbers from a given distribution}
The idea is to sample from the probability distribution $\rho_n(x) = c x^n$ in
$[0, 1]$.  First, using the normalization condition we can find out what $c$
should be:
\begin{equation}
    1 = \int_{0}^{1} cx^n \, dx = \frac{c}{n + 1} \implies c = n + 1.
\end{equation}
Then, we find the expression of the associated cumulative density function:
\begin{equation}
    F_n(x) = (n + 1) \int_{0}^{x} y^n \, dy = x^{n + 1},
\end{equation}
and invert it:
\begin{equation}
    p = x^{n + 1} \implies x = p^{1 / (n + 1)}.
\end{equation}

So, inside the code \texttt{A02a\_inversion\_method.c} I sample a random
\texttt{double} from a uniform distribution between $0$ and $1$ using
\texttt{drand48()}, and I raise it to the power of $1 / (n + 1)$ to get $x$:

\lstinputlisting[
    firstline=25, lastline=27, language=C
]{../src/A02a_inversion_method.c}

A histogram of \num{100000} points sampled from $\rho$ with $n = 3$ is displayed
in \figref{fig:A02a_3}.

\begin{figure}
    \centering
    \includesvg[inkscapelatex=false]{img/A02a_3.svg}
    \caption{histogram of \num{100000} points sampled from the probability
        distribution $4 x^3$ in $[0, 1]$.}
    \label{fig:A02a_3}
\end{figure}


\end{document}
